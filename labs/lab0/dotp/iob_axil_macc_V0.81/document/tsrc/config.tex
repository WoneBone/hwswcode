
The following tables describe the IP core configuration. The core may be configured using macros or parameters:

\begin{description}
    \item \textbf{'M'} Macro: a Verilog macro or \texttt{define} directive is used to include or exclude code segments, to create core configurations that are valid for all instances of the core.
    \item \textbf{'P'} Parameter: a Verilog parameter is passed to each instance of the core and defines the configuration of that particular instance.
\end{description}

\begin{table}[H]
  \centering
  \begin{tabularx}{\textwidth}{|l|c|c|c|c|X|}

    \hline
    \rowcolor{iob-green}
    {\bf Configuration} & {\bf Type} & {\bf Min} & {\bf Typical} & {\bf Max} & {\bf Description} \\ \hline \hline

    \input general_operation_confs_tab

  \end{tabularx}
  \caption{General operation group}
  \label{general_operation_confs_tab:is}
\end{table}

There are also constants that are used in the core in order to improve the readability
of the code and should not be changed. They are defined as presented in the list below:
\input constants
\clearpage